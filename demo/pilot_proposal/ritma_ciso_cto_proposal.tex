% Ritma CISO/CTO Pilot Proposal
% Author: Umesh (Ritma)

\documentclass[11pt]{article}
\usepackage[margin=1in]{geometry}
\usepackage{hyperref}
\usepackage{enumitem}

\setlist[itemize]{noitemsep,topsep=2pt,leftmargin=1.5em}

\hypersetup{
  colorlinks=true,
  linkcolor=black,
  urlcolor=blue,
  pdftitle={Ritma CISO/CTO Pilot Proposal}
}

\begin{document}

% FRONT PAGE – 1-PAGE DIRECT PITCH
\begin{center}
  {\LARGE Ritma Pilot Proposal}\\[4pt]
  {\large CISO / CTO Briefing}\\[8pt]
  {Prepared by Umesh (Ritma)}\\[2pt]
  {Version: Pilot-ready, December 2025}
\end{center}

\vspace{1em}

Ritma is aimed at teams that already have logging, monitoring, and policy
enforcement in place, but who need to be able to \emph{prove}, to themselves,
their board, auditors, and regulators, that these systems behaved as claimed at
specific points in time.

\medskip

\noindent\textbf{Ritma: Cryptographic Evidence Fabric Around Your Existing Stack}

\begin{itemize}
  \item Ritma adds a \emph{cryptographically provable audit fabric} around your existing SIEM / IAM / policy stack, without forcing you to replace tools that already work.
  \item Pilot is sidecar-only: we observe and prove, but do not sit inline on critical production paths or become a new availability risk.
  \item You get hash-chained logs, truth snapshots ("Git commits for reality"), and verifiable evidence packages that can be checked independently from the Ritma team.
  \item Commercially: a 10-day free guided demo, then an optional paid pilot with clear success criteria, decision gate, and no lock-in if it does not meet your bar.
\end{itemize}

\newpage

% CHAPTER 1 – PROBLEM & CONTEXT
\section{Chapter 1: Problem \\ Why Evidence Is Broken Today}

Modern organizations already collect vast quantities of logs and metrics, but
when regulators or incident response teams ask hard questions, those logs are
often not enough to give a confident, defensible answer. This chapter explains
why the status quo around evidence is fragile.

\begin{itemize}
  \item Logs are cheap to generate and easy to silently lose or rewrite.
  \item Compliance evidence is manual, brittle, and rarely cryptographically provable.
  \item Policy and access decisions are hard to reconstruct and defend months later.
  \item Evidence pipelines from raw logs to final audit decks are often undocumented, making it hard to show auditors exactly how results were produced.
  \item Responsibility for end-to-end evidence integrity is scattered across teams; no single system is accountable for the full chain of custody.
\end{itemize}

% CHAPTER 2 – ARCHITECTURE & DATA FLOWS
\section{Chapter 2: Architecture, Data Flows, and Evidence Model}

This chapter describes how Ritma is deployed in a pilot, what components are
involved, and what data flows through the system. The goal is to make it
obvious where Ritma sits in relation to your existing SIEM, IAM, and policy
stack.

\begin{itemize}
  \item Components: \texttt{utld} daemon (truth layer) and \texttt{utl\_cli} (CLI client/reporting).
  \item Storage: \texttt{dig\_index.sqlite}, \texttt{decision\_events.jsonl}, \texttt{compliance\_index.jsonl}.
  \item All data remains on your infrastructure; payloads can be redacted or hashed.
  \item Integration pattern: applications and policy engines send structured events to Ritma over a local socket; Ritma never needs direct access to production databases or secrets.
  \item Network stance: in a pilot, Ritma does not require outbound internet access; all state is local so you can test it entirely inside your own security boundary.
\end{itemize}

% CHAPTER 3 – SECURITY & THREAT MODEL
\section{Chapter 3: Security Posture, Threat Model, and Failure Modes}

From a security and risk perspective, the key questions are: what happens if a
Ritma node is compromised, if its storage is corrupted, or if keys leak? This
chapter outlines our assumptions and how integrity and blast radius are
managed in each scenario.

\begin{itemize}
  \item We assume baseline OS/IAM controls; Ritma focuses on integrity and provability of evidence.
  \item Node compromise or DB corruption is detectable via hash-chains and truth snapshot verification.
  \item Pilot mode is non-inline: if Ritma is down, production continues; you only lose new evidence, not availability.
  \item Ritma's own operations are logged and can be included in evidence packages, so you can see when policies were changed, when evidence was exported, and by whom.
  \item Signing and verification keys are configurable to use your existing secret-management or KMS setup in a production deployment.
\end{itemize}

% CHAPTER 4 – PILOT OFFER
\section{Chapter 4: Pilot Offer \\ 10-Day Demo to Paid Pilot}

This chapter sets out the commercial and operational path: a low-friction
10-day demo, followed by an optional, tightly scoped paid pilot with clear
success criteria and an explicit yes/no decision point.

\begin{itemize}
  \item 10-day free demo on staging/non-prod with 1--2 real workflows.
  \item If useful: 60--90 day paid pilot with defined control objectives and SLOs for evidence.
  \item Optional co-enhancement track for reports, integrations, and deeper proofs.
  \item Clear roles: we expect a CISO/CTO sponsor, a security engineer, and an application owner to be named for the pilot so decisions are fast and grounded.
  \item Commercial terms, data boundaries, and success criteria are written down up front so there are no surprises at the end of the pilot.
\end{itemize}

% CHAPTER 5 – EVIDENCE, EVALUATION, NEXT STEPS
\section{Chapter 5: Evidence, Evaluation, and Next Steps}

Finally, we outline how you can independently verify Ritma's outputs, how you
should evaluate the pilot, and what concrete steps lead from demo to a
go/no-go decision on a broader deployment.

\begin{itemize}
  \item You can independently run hash and signature verification on logs and evidence packages, without needing any secret knowledge from the Ritma team.
  \item Evaluation questions: can we answer "who knew what, when?" for a real incident, and can we satisfy at least one real audit or control objective with Ritma-derived evidence?
  \item Next steps: agree pilot scope, run the 10-day demo, review the resulting evidence and reports together, then decide on a paid pilot.
  \item If the answer is ``no'', the pilot ends with a clear, documented reason; if the answer is ``yes'', we transition to production planning with explicit timelines.
\end{itemize}

\end{document}
